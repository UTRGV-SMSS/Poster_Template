\documentclass{beamer}

%size given in centimeters appropriate for SPIE
\usepackage[size=custom,width=114.3,height=91.44,scale=1]{beamerposter}
\setbeamersize{text margin left=1in, text margin right=1in}
\usepackage{utrgv_light_poster}

% for urls
\usepackage{hyperref}

% for numbered bibliography
\setbeamertemplate{bibliography item}[text]

% useful packages
\usepackage{graphicx}

%-- Header and footer information ----------------------------------
\title{UTRGV Light Poster Example and Template}
\author{Jane Doe, Jaime Escalante}
\institute{School of Mathematical and Statistical Sciences}
%-------------------------------------------------------------------


%%==============================================================================
%==the poster content==========================================================
%==============================================================================
\begin{document}\large
%--the poster is one beamer frame, so we have to start with:
\begin{frame}[t]
%--to seperate the poster in columns we can use the columns environment
\begin{columns}[t] % the [t] options aligns the columns content at the top



%--the left column-------------------------------------------------------------
\begin{column}{0.28\paperwidth}% the right size for a 3-column layout
%--abstract block--------------------------------------------------------------
   \begin{alertblock}{Objectives}
     This is a multi-column poster template that can be used for SPIE and other
     conferences. As written this template can be used for columns of multiple
     sizes or columns within columns. This latex file must be compiled several
     times to get the correct edges. For a nice example of a poster see
     \url{https://teamwork.jacobs-university.de:8443/confluence/display/CoPandBiG/LaTeX+Poster}.
     The font used is Raleway semibold.  This template can also be rewritten to use the simpler style used in the dark poster template, but you must keep the outer frame defined just below the \texttt{\textbackslash begin \{document\}} command.  As always you can use an internet search for inspiration. Just search for \textit{beamerposter}.


   \end{alertblock}
   \vskip2ex
\end{column}


%===big rightcolumn=============================================================
\begin{column}{0.60\paperwidth} %thats the big right column
%--Examples block---------------------------------------------------------------
 \begin{block}{Examples}
   Use this to have text and figures that
   span approximately 2/3 of the poster.
   Use this to have text and figures that
   span approximately 2/3 of the poster.
 \end{block}
 \vskip2ex


%===two right columns===========================================================
\begin{columns}[t,totalwidth=0.60\paperwidth]
 \begin{column}{0.28\paperwidth}
   Use this to have text and figures that
   span approximately 1/3 of the poster.
   Use this to have text and figures that
   span approximately 1/3 of the poster.
   \begin{equation}
     c_n = \frac{1}{T} \int_{-\infty}^{\infty} f(x) dx
   \end{equation}

   \begin{block}{More Examples}
    Use this to have text and figures that
    span approximately 1/3 of the poster.
    Use this to have text and figures that
    span approximately 1/3 of the poster.
   \end{block}
 \end{column}


 \begin{column}{0.28\paperwidth}
   Use this to have text and figures that
   span approximately 1/3 of the poster.
   Here is math text:


 \begin{alertblock}{Conclusion}
   Here is another block that you can use for the conclusion.
 \end{alertblock}

 \end{column}
\end{columns}
\end{column}

\end{columns}



\end{frame}
\end{document}
